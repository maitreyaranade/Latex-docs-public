\section{Introduction}    
A sequential circuit is specified by a time sequence of inputs, outputs, and internal states. It consists of a combinational circuit to which memory elements are connected to form a feedback path.

\begin{highlight}
    Insert Figure 5.1 Block diagram of sequential circuit
\end{highlight}

\par The storage elements are devices capable of storing binary information. The binary information stored in the memory elements determines the "state" of the sequential circuit. The sequential circuits receive binary information from the external inputs that, together with the present state of the storage elements, determine the binary value of the outputs. The next state of the storage elements is also a function of external inputs and the present state. 

\section{Types of sequential circuits}
\begin{itemize}
    \item Synchronous sequential circuits: Circuits whose behavior can be defined from the knowledge of it's signals at discrete instances of time.    
    \item Asynchronous sequential circuits: Circuits whose behavior depends on the input signals at any instance of time and the order 
    in which the input changes.
\end{itemize}

In synchronous sequential circuits, synchronization is achieved by a timing device called as \textit{clock generator}, which provides a periodic signal  in the form of train of \textit{clock pulses}. The clock signal is commonly denoted by the identifiers \textit{\textbf{clock}} or \textit{\textbf{clk}}. The clock pulses are distributed throughout the system.

\par Synchronous sequential circuits, that use clock pulses to control storage elements are called \textit{clocked sequential circuits}. The storage elements (memory) used in clocked sequential circuits are called Flip-Flops. A Flip-Flop is a binary storage device capable of storing one bit of information.

\begin{highlight}
    Insert fig 5.2 synchronous sequential circuits
\end{highlight}

A storage element in a digital circuit can maintain a binary state indefinitely (as long as power is delivered to the circuit), until directed by an input signal to switch states.

\section{Storage Elements: Latches}
Storage elements that operate with signal levels (rather than signal transitions) are referred to as \textit{\textbf{latches}} \& those controlled by a clock transition are \textit{\textbf{flip-flops}}. Latches are said to be level sensitive devices; flip-flops are edge-sensitive devices. The two types of storage elements are related because latches are the basic circuits from which all flip-flops are constructed.

\subsection{SR Latch (Set Reset Latch)}
The SR latch is a circuit with two cross-coupled NOR gates or two cross-coupled NAND gates, and two inputs labeled for set(S) and reset(R). The latch has two useful states. When output is tied high, the latch is said to be in the set state. When output is tied low, the latch is said to be in the reset state. Outputs Q and Q' are normally the complement of each other.

\begin{highlight}
    Insert Figure 5.3 5.4 SR Latch implementation with NOR gates and NAND gates.
\end{highlight}

SR latch with two cross-coupled NOR gates:
\begin{itemize}
    \item Setting both inputs to 1 is forbidden as it might lead to unpredictable/ "metastable" state.
    \item When both inputs are equal to 1 at the same time, a condition in which both outputs are equal to 0 (rather than be mutually complementary) occurs.
    \item If both inputs are then switched to 0 simultaneously, the device will enter an unpredictable or undefined state or a metastable state.
    \item When inputs are applied, the resulting (next) state is a function of inputs as well as present state of the latch.
\end{itemize}

\subsubsection{SR latch with control input}
The operation of the basic SR latch can be modified by providing an additional control input signal that controls when the state of the latch can be changed by determining whether S and R (or S' and R') can affect the circuit.

\par The control input "En" acts as an enable signal for the other two inputs. The outputs of the NAND gates stay at the logic-1 level as long as the enable signal remains at 0. This is the quiescent
condition for the SR latch. When the enable input goes to 1, information from the S or R
input is allowed to affect the latch. The

\begin{highlight}
    Insert FIGURE 5.5 SR latch with control input
\end{highlight}