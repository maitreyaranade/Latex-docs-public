\section{Boolean algebra}
THe most Common postulates used to formulate various algebraic structures are: 
\begin{enumerate}
    \item Closure: A set S is closed with respect to a binary operator if, for every pair of elements of S, the binary operator specifies a rule for obtaining a unique \& element of S.
    \item Associative law: \[(x*y)*z = x*(y*z)\] for all \(x,y,z \varepsilon S\)  
    \item Commutative law: \[x*y = y*x\] for all \(x,y\varepsilon S\)
    \item Identity element: if there exists an element \(e \varepsilon S\) with \[e*x = x*e = x\] for every \(x \varepsilon S\)
    \item Inverse: A set S having the identity element e - is said to have an inverse when , \[x*y=e\]
    \item Distributive law: \[x*(y.z) = (x*y).(x*z) \]
\end{enumerate}

\section{Basic theorems \& Properties of boolean algebra}
\subsection{Duality} If the operators and the elements are interchanged, every algebraic expression deducible from the postulates of boolean algebra remains unchanged. 
\subsection{Basic theorems}
- Insert table 2.1 
\subsection{Operator Precedence}
\begin{enumerate}
    \item Parentheses
    \item NOT
    \item AND
    \item OR
\end{enumerate}

\section{Boolean functions}
A boolean function is described by an algebraic expression consisting of binary variables, the constants (0 \& 1) and the logic operation symbols. 
\[eg. F_1 = x + y'z \] 

\par A boolean a n function expresses the logical relationship between binary variables \& is evaluated by determining the binary value of the expression for all possible values of the variables. 

\subsection{Truth table} is a boolean function can be represented in a 
truth table. Truth table enlists all the possible combinations \(2^n\) of the variables (n), and the corresponding output value of boolean function.\\
A boolean function can be transformed into a circuit diagram composing logic gates from algebraic expression which is also known as schematic.

\section{Canonical \& standard forms}
revisit 
Insest logic gates one pager 
Fig. 205

\section{Integrated Circuits}
An integrated circuit (IC) is fabricated on a die of a silicon semiconductor crystal, called a chip, containing the electronic components for constructing digital gates.

\subsection{Levels of Integration}
Digital ICs are categorized according to the complexity of their circuits.
\begin{itemize}
    \item SSI (Small Scale Integration): Number of gates are less than 10.
    \item MSI (Medium Scale integration): Number of gates are between 10 to 1000 Gates.
    \item LSI (Large Scale Integration): Number of gates are in thousands ex. Processors, memory chips \& programmable logic devices.
    \item VLSI (Very Large Scale Integrations): Number of gates are in millions ex. complex microcomputer chips. 
\end{itemize} 

\section{Computer Aided Design of VLSI circuits}
Integrated circuits having sub-micron geometric features are manufactured optically by projecting light onto silicon wafers. The design of digital systems with VLSI circuits containing millions of gates is an enormous task. Only with the help of CAD tools, this task is made possible considering system complexity.\\ 
EDA (Electronic Design Automation) automates multiple steps of design phases of ICs. Typical design flow, 

\begin{enumerate}
    \item Design entry: Schematic / HDL based model 
    \item phyical realisation 
    \begin{enumerate}
        \item ASIC 
        \item FPGA 
        \item PLD 
        \item Full Custom IC
    \end{enumerate}
    \item Hardware fabrication with CAD \& EDA (Schematic entry)
\end{enumerate}