\section{Number conversion. Revise, watch youtube videos}    
\begin{itemize}
    \item Binary Number representation
    \item Binary Number conversion
    \item Octal Hex Number representation        
    \item Complement of numbers
    \item Signed Binary Number representation
    \item BCD code representation
    \item Gray code representation
\end{itemize}

\section{Binary storage and registers}
The binary information in a digital computer must have a physical existence in some medium for storing individual bits. A binary cell is a device that possesses two stable states and is capable of storing one bit (0 or 1) of information.

\subsection{Registers} 
A register is a group of binary cells. A register with n cells can store any discrete quantity of information that contains n bits.  The state of a register is an n‐tuple of 1’s and 0’s, with each bit designating the state of one cell in the register. For example, a 16 bit register with the following binary content: 1100001111001001. A register with n cells can be in one of $2^{n}$ possible states.

\subsection{Register Transfer}
A digital system is characterized by its registers and the components that perform data processing. In digital systems, a register transfer operation is a basic operation that consists of a transfer of binary information from one set of registers into another set of registers.The transfer may be direct, from one register to another, or may pass through data‐processing circuits to perform an operation.



