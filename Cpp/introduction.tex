\subsection{History of Computing}

Manual Entering code in binary 
Assembly language 

C is a procedural programming language. It was initially developed by Dennis Ritchie in the year 1972. It was mainly developed as a system programming language to write an operating system. The main features of the C language include: 
\begin{itemize}
    \item Portability (Machine Specific)
    \item Requires less lines of code than assembly language
    \item Procedural programming
    \item Middle level language   
    \begin{itemize}
        \item Direct access to memory through Pointers
        \item Bit manipulation using bitwise Operators
        \item Writing assembly code within C code
    \end{itemize}
    \item Popular choice for system level apps
    \item Wide variety of built in functions, standard libraries and header files.
\end{itemize}

High level vs Low level language
Degree of abstraction 
High level: Less efforts to users ex. COBOL, FORTRAN, C++, Pascal etc.
Low level: More efforts to users ex. Assembly
Middle level language: ex. C 

Preprocessor: replaces text (starting with \#) with actual content before compilation. Output of preprocessing is expanded source code.

Preprocessor directive: ex. \#include<stdio.h>
stdio.h: standard input output file 
\begin{itemize}
    \item header files: .h extensions  
    \item Contains declarations of functions like printf, scanf etc.
\end{itemize}

\begin{itemize}
    \item Variables
    \item Operators
    \item Conditionals and Loops
    \item Functions
    \item Recursion
    \item Pointers and arrays
    \item Structure and union
\end{itemize}