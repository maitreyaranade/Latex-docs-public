\subsection{DDR2 RAM}
After the first generation, the second generation of DDR RAM is DDR2 RAM. DDR2 SDRAM superseded the original DDR1 SDRAM specification, and was superseded by DDR3 SDRAM when launched in 2007. The maximum capacity on commercially available DDR2 DIMMs is 8GB, but chipset support and availability for those DIMMs is sparse and more common 2GB per DIMM are used. In case of DDR2 RAM, it is operated at 1.8 V instead of 2.5 V unlike the DDR1 RAM. The internal RAM clock frequency is same as the previous generation. Instead the data rate is doubled compared to the first generation which was achieved by increasing the number of bits that are being pre-fetched during each cycle. In case of this DDR2 RAM instead of 2 bits, 4 bits are pre-fetched during each cycle. In other words, the internal bus width of DDR2 RAM has been doubled when compared with DDR1.

\par For \eg suppose if the input output bus is 64 bits wide, then the internal bus width of this RAM will be equal to 128 bits. So, in this way, in a single cycle, this RAM can handle double amount of data. To handle the same amount of data, the clock frequency of this input output bus should be get doubled. Suppose DDR2 RAM, is operated at 100 MHz internal clock frequency then the input output bus should have the clock frequency of 200 MHz. And in case of this DDR RAM, as data is transferred both during rising and falling edge, so the data rate will be doubled compared to the clock frequency, that is 400 mega transfer per second. Suppose if DDR2 RAM is operated at 400 MHz clock frequency, then the data rate will be equal to 800 mega transfer per second. And in terms of DDR terminology, it can be written as DDR2-800 or PC2-6400.

\subsection{DDR3 RAM}

After the second generation, the third generation of DDR RAM is the DDR3 RAM. DDR3 SDRAM superseded the original DDR2 SDRAM specification, and was superseded by DDR4 SDRAM when launched in 2014. The DDR3 standard permits DRAM chip capacities of up to 8 gibibits (Gibit), and up to four ranks of 64 bits each for a total maximum of 16 gibibytes (GiB) per DDR3 DIMM. In case of this DDR3 RAM, the voltage is further reduced from 1.8V to the 1.5V. The internal clock frequency of DDR3 RAM is slightly improved compared to DDR2. But the data rate that you can achieve with the same frequency has been doubled as compared to the DDR2. In case of DDR3 RAM, the number of bits that is being pre-fetched has been further increased from 4 bits to the 8 bits. In other words, the internal data bus width of RAM has been increased 2 times compared to DDR2 and 4 times to DDR1.
\par For \eg suppose if the internal clock frequency is 100 MHz, then to match the data rate, the input output bus should be get operated at the 4 times the clock frequency that is 400 MHz. And the transfer rate will be 800 mega transfer per second. For \eg DDR3-800 followed by PC3-6400 on any DDR3 RAM, means that the clock frequency of this RAM is 400 MHz and the maximum transfer rate which can be achieved is 800 mega transfer per second. Maximum bandwidth of the RAM is 6400 Megabytes per second.

\subsection{DDR4 RAM}
DDR4 SDRAM is the abbreviation for 'Double Data Rate fourth generation synchronous dynamic random-access memory', the latest variant of memory in computing. DDR4 is able to achieve higher speed and efficiency thanks to increased transfer rates and decreased voltage. The primary advantages of DDR4 over its predecessor, DDR3, include higher module density and lower voltage requirements, coupled with higher data rate transfer speeds. The DDR4 standard allows for DIMMs of up to 64 GiB in capacity, compared to DDR3's maximum of 16 GiB per DIMM.

\par After the third generation, the fourth generation of DDR RAM is DDR4 RAM. DDR4 SDRAM superseded the original DDR3 SDRAM specification, and was superseded by DDR5 SDRAM when launched in 2020.
The DDR4 standard allows for DIMMs of up to 64 GiB in capacity, compared to DDR3's maximum of 16 GiB per DIMM. In case of this DDR4 RAM, the operating voltage has been further reduced from
1.5V to the 1.2V. The number of bits that are being pre-fetched is same as DDR3 i.e. 8 bits per cycle. In case of this DDR4 RAM, the internal clock frequency of the RAM has been increased.
For \eg suppose if you are operating at 400 MHz then the clock frequency of the input output bus should be 4 times, that means 1600 MHz. The transfer rate will be equal to 3200 Mega transfer per second. Module terminology \eg DDR4-3200 followed by PC4-25600. 25600 is the speed in terms of Megabytes per second.

\begin{itemize}
    \item \textbf{Single channel mode:} In single channel mode, the physical RAM, uses the usual input output bus width (64 bits).
    \item \textbf{Dual channel mode} In dual channel mode, the same physical RAM, uses twice the input output bus width to effectively achieve twice the data rate. Suppose, there is an 8 GB DDR4 RAM running in single channel mode and two 4 GB DDR4 RAMs running in dual channel mode, then the bandwidth that can be achieved with two 4 GB of DDR4 RAM will be better as compared to the single channel 8GB of DDR4 RAM.
\end{itemize}

\subsection{Application specific DDR versions}
A compact version of DIMM module is known as Small Outline DIMM Modules (SO-DIMM). Another version of dynamic RAMs which are used inside the mobile or smartphones are known as the mobile DDR or Low Power DDR. Low Power DDR RAMs are also having different generations. Starting from LPDDR1 up to the LPDDR4. LPDDR RAMs are optimised for the low power consumption. Another specially catered version of DDR RAMs which is used for graphics cards is known as the graphics DDR or GDDR. As this Graphical DDR is used for the multimedia applications, the data handling is quite extensive. Hence, GDDR RAMs have larger bandwidth compared to usual DDRs. 

\subsection{DDR Packaging}
The older generations of DRAMs were available in the Dual Inline Package(DIP). Then after the next generation of RAMs were available in the Single In-Line Modules (SIMM). In Single In-Line module, the memory chips are soldered onto the one PCB, and the pins are available on the single side of the PCB. And that is a reason, it is known as the Single In-Line Modules. Single In-line module can provide data bus width of 32 bits. But suppose if you want 64 bits of the data bus, then you need to connect the two single In-line modules in the parallel.

\par After the next generation of RAMs were available in the Dual In-Line Module or DIMM. In Dual In-Line Module, it is possible to have 64 bits wide data bus. Also, the pins are available both in front as well as the back of the PCB. And that is a reason, it is known as the Dual In-Line Module.

\par All the DDR generations have a different number of pins as well as the different operating voltage. Hence, all the four generation of RAMs are not either forward or backward compatible. So, suppose a motherboard supporting DDR3 RAM, will not support either DDR2 or DDR4 RAM.

\subsection{Xilinx DDR MIG Controller IP}

The Memory Interface Generator (MIG) generates DDR4 SDRAM, DDR3 SDRAM, DDRII SRAM, DDR SDRAM, DDR2 SDRAM, QDRII SRAM, and RLDRAM II interfaces for various Xilinx FPGAs. The tool takes inputs such as the memory interface type, FPGA family, FPGA devices, frequencies, data width, memory mode register values, and so forth, from the user through a graphical user interface (GUI). The
tool generates RTL, SDC, UCF, and document files as output. RTL or EDIF (EDIF is created after running a script file, where the script file is a tool output) files can be integrated with other design files.

MIG is a tool used to generate memory interfaces for Xilinx FPGAs. MIG generates Verilog or VHDL RTL design files, user constraints files (UCF), and script files. The script files are used to run simulations, synthesis, map, and par for the selected configuration.