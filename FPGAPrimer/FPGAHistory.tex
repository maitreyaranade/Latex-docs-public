\subsection{History of Programmable Logic}


\begin{description}
\item [TTL(Transistor Transistor Logic) logic design] Implementng logic by using basic logic funtions available on separate chips/ ICs (ex: Texas Instruments 7400 device family) on a breadboard. Methodology: 
    \begin{itemize}
    \item Creating truth table.
    \item Cration Karnaugh map.
    \item Generate logical expression.
    \item Final logic implementation.
    \end{itemize}

\item [Programamble logic] 
    \begin{description}
    \item  [Programamble Array Logic(PAL)] Simplest implementation of programmable logic. Logic gates and registers fixed. 
    Programmable sum of products array and output control. Floating-gate transistors at array crossings set to never conduct after applying programming voltages.
    \\ Put image here

    \item [Programamble Logic Devices (PLD)] Arrange multiple PAL arrays in a single device. It is comprised of: i. Variable product term distribution ii. Programmable macrocells

    Programmable macrocells:
    Generated programmable output from sum of products. Provided feedback (using output pin as input)

    \end{description}

\item [Complex Programamble Logic Devices (CPLD)] Combine multiple PLDs(logic blocks) in a single device with programmable interconnect and I/O.

Put image here:

CPLD logic block or Logic Array Blocks(LAB):

Contain multiple macrocells (typically 4 to 20)
Local programmable interconnect like a PLD.




\item [Other architectures]

    \begin{description}
    \item [Programmable Interconnect Array(PI or PIA)] Similar to PAL programming technology. Global routing connects any signal to any destination in device. Programmed wih EPROM,EEPROM or flash technology.

    \item [I/O control blocks] Introduction in CPLDs. Seprated from logic by PI. I/O specific logic provides control, more features. Tri-state buffer control to enable input, outputs, or bidirectional on any I/O pin.

    \item [In-System Programming (ISP) with JTAG] Simple 4 or 5 wire serial interface. Shifts data through one or more devices on a board (JTAG chain). Used for device self test or ISP.
PLD Hardware generates EEPROM programming voltages controlled by JTAG interface.

    \end{description}


\end{description}
