\section{AXI}
\subsection{Protocol Overview}
Xilinx adopted the Advanced eXtensible Interface (AXI) protocol for Intellectual Property (IP)

There are three types of AXI4 interfaces: 
\begin{enumerate}
    \item AXI4: For high-performance memory-mapped requirements
    \item AXI4-Lite: For simple, low-throughput memory-mapped communication (for example, to and from control and status registers)
    \item AXI4-Stream: For high-speed streaming data.
\end{enumerate}

\subsubsection{Summary of AXI4 Benefits}
\begin{enumerate}
    \item Productivity: By standardizing on the AXI interface, developers need to learn only a single protocol for IP.
    \item Flexibility: Providing the right protocol for the application: 
    \begin{enumerate}
        \item AXI4 is for memory-mapped interfaces and allows high throughput bursts of up to 256 data transfer cycles with just a single address phase. 
        \item AXI4-Lite is a light-weight, single transaction memory-mapped interface. It has a small logic footprint and is a simple interface to work with both in design and usage.
        \item AXI4-Stream removes the requirement for an address phase altogether and allows unlimited data burst size. AXI4-Stream interfaces and transfers do not have address phases and are therefore not considered to be memory-mapped. 
    \end{enumerate}
    \item Availability: By moving to an industry-standard, you have access not only to the Vivado IP Catalog, but also to a worldwide community of ARM partners.
    \begin{enumerate} 
        \item Many IP providers support the AXI protocol.
        \item A robust collection of third-party AXI tool vendors is available that provide many verification, system development, and performance characterization tools. As you begin developing higher performance AXI-based systems, the availability of these tools is essential.
    \end{enumerate}
\end{enumerate}


\subsubsection{How AXI Works}
\begin{enumerate}
    \item The AXI specifications describe an interface between a single AXI master and AXI slave, representing IP cores that exchange information with each other. Multiple memory-mapped AXI masters and slaves can be connected together using AXI infrastructure IP blocks
    \item Both AXI4 and AXI4-Lite interfaces consist of five different channels: 
    \begin{enumerate}
        \item Read Address Channel
        \item Write Address Channel
        \item Read Data Channel
        \item Write Data Channel
        \item Write Response Channel
    \end{enumerate}
    \item Data can move in both directions between the master and slave simultaneously, and data transfer sizes can vary. The limit in AXI4 is a burst transaction of up to 256 data transfers (Requires a single address and then bursts up to 256 words of data). AXI4-Lite allows only one data transfer per transaction.
    \item AXI4 Read Transaction:
        \begin{figure}[H]
        \begin{center}
        \includegraphics[width=4in]{images/AXIREAD.png}
        \caption{AXI4 Read Transaction}
        \label{AXIREAD}
        \end{center}
        \end{figure} 
    \item AXI4 Write Transaction:
        \begin{figure}[H]
        \begin{center}
        \includegraphics[width=4in]{images/AXIWRITE.png}
        \caption{AXI4 Write Transaction}
        \label{AXIWRITE}
        \end{center}
        \end{figure} 
    \item At a hardware level, AXI4 allows systems to be built with a different clock for each AXI master-slave pair. In addition, the AXI4 protocol allows the insertion of register slices (often called pipeline stages) to aid in timing closure.
    \item AXI4-Lite is similar to AXI4 with some exceptions: The most notable exception is that bursting is not supported.
    \item The AXI4-Stream protocol defines a single channel for transmission of streaming data. The AXI4-Stream channel models the write data channel of AXI4. Unlike AXI4, AXI4-Stream interfaces can burst an unlimited amount of data.
\end{enumerate}
