\subsection{DDR RAM} 
In case of the next generation of the SDR DRAM (Single Data Rate DRAM or synchronous DRAM), the data is transferred twice during the clock cycle. First, during the positive edge and secondly, during the negative edge of the clock cycle. And that is why this generation of the SDRAM is known as the \textbf{Double Data Rate} or \textbf{DDR SDRAM}. 

\par There are different generations of DDR RAM ranging from the DDR1 up to the DDR4 which is considered to be the latest. The first generation of DDR RAM is known as the DDR1 RAM. As compared to the SDR SDRAM, the voltage levels has been reduced from 3.3V to the 2.5V.

\par There are a total two types of frequencies associated with DRAM:

\begin{description}
	\item[Input output clock frequency] is the frequency at which the data is being transferred between the RAM and the memory controller.
	\item[RAM Internal clock frequency] of the RAM is the frequency which is being used by the RAM for the internal operations. 
\end{description}

In case of SDRAM, input output clock frequency and the internal clock frequency of the RAM are same. For \eg PC-100 specification on the SDRAM module means that the input output clock frequency is 100 mega transfers per second and if the data bus is 64 bit wide, then the data rate in terms of the bits per second will be 100 MHz into 64 bits. Which is 800 Megabytes per second.

\par In case of DDR RAM, the data is being transferred both during the rising as well as the falling edge of the clock cycle. Hence, in a single clock cycle, instead of a single bit, 2 bits are pre-fetched which is known as the \textbf{2 bit pre-fetch}. In case of DDR1 RAM, the internal clock frequency, as well as the input output bus clock frequency, are same. Generally, DDR1 RAM is operated in the range of 133 MHz up to 200 MHz. But if you see the data rate at the input output bus, it will be double compared to the clock frequency. In case of DDR1 RAM, the data is transferred both during rising as well as the falling edge. For \eg suppose if DDR1 RAM is operated at 133 MHz then the data rate will be 266 Mega transfer per second. If the bus frequency is 200 MHz then the data transfer rate will be 400 Mega transfer per second. And if the input output bus is 64 bits wide, then the data rate will be 3200 Megabytes per second.

\par Nowadays, DDR RAMs are generally denoted by the term DDR followed by the transfer rate of this RAM. For \eg a DDR1 module or a DDR1 stick, most probably will have a specification like PC-3200. It means that the maximum speed or the maximum bandwidth which can be achieved by this DDR1 RAM is 3200 Megabytes per second.