\section{Flash Memory} 
Flash memory is an electronic (solid\textendash state) non\textendash volatile computer memory storage medium that can be electrically erased and reprogrammed. The two main types of flash memory are named after the NAND and NOR logic gates. The individual flash memory cells, consisting of floating\textendash gate MOSFETs (floating\textendash gate metal\textendash oxide\textendash semiconductor field\textendash effect transistors), exhibit internal characteristics similar to those of the corresponding gates. While EPROMs had to be completely erased before being rewritten, NAND\textendash type flash memory may be erased, written and read in blocks (or pages) which are generally much smaller than the entire device. NOR\textendash type flash allows a single machine word (byte) to be written to an erased location or read independently. A flash memory device typically consists of one or more flash memory chips (each holding many flash memory cells) along with a separate flash memory controller chip. The NAND type is found primarily in memory cards, USB flash drives, solid\textendash state drives (those produced in 2009 or later), and similar products, for general storage and transfer of data. NAND or NOR flash memory is also often used to store configuration data in numerous digital products, a task previously made possible by EEPROM or battery\textendash powered static RAM. 

\textbf{Serial Flash} Serial flash is a small, low\textendash power flash memory that provides only serial access to the data \textendash  rather than addressing individual bytes, the user reads or writes large contiguous groups of bytes in the address space serially. Serial Peripheral Interface Bus (SPI) is a typical protocol for accessing the device. When incorporated into an embedded system, serial flash requires fewer wires on the PCB than parallel flash memories, since it transmits and receives data one bit at a time. This may permit a reduction in board space, power consumption, and total system cost. A flash memory controller (or flash controller) manages data stored on flash memory and communicates with a computer or electronic device. Flash memory controllers can be designed for operating in low duty\textendash cycle environments like SD cards, Compact Flash cards, or other similar media.

\subsection{QSPI Flash}
AXI Quad SPI LogiCORE IP AXI Quad Serial Peripheral Interface (SPI) core connects the AXI4 interface to those SPI slave devices that support the Standard, Dual, or Quad SPI protocol instruction set. This core provides a serial interface to SPI slave devices. The Dual/Quad SPI is an enhancement to the standard SPI protocol (described in the Motorola M68HC11 data sheet) and provides a simple method for data exchange between a master and a slave.

Configurable SPI modes:
\begin{itemize}
  \item Standard SPI mode
  \item Dual SPI mode
  \item Quad SPI mode
  \item Programmable SPI clock 
\end{itemize} 
