\documentclass[12pt, a4paper]{article}
% Verilog code formatting ref: https://tex.stackexchange.com/questions/377122/typesetting-for-a-verilog-lstinput
\usepackage{xcolor}
\usepackage{listings}
\definecolor{vgreen}{RGB}{104,180,104}
\definecolor{vblue}{RGB}{49,49,255}
\definecolor{vorange}{RGB}{255,143,102}

\lstdefinestyle{verilog-style}
{
    language=Verilog,    
    breaklines=true,    
    basicstyle=\small\ttfamily,
    keywordstyle=\color{vblue},
    identifierstyle=\color{black},
    commentstyle=\color{vgreen},
    numbers=left,
    numberstyle=\tiny\color{black},
    numbersep=10pt,
    tabsize=8,
    moredelim=*[s][\colorIndex]{[}{]},
    literate=*{:}{:}1
}

\definecolor{mGreen}{rgb}{0,0.6,0}
\definecolor{mGray}{rgb}{0.5,0.5,0.5}
\definecolor{mPurple}{rgb}{0.58,0,0.82}
\definecolor{backgroundColour}{rgb}{0.95,0.95,0.92}

\lstdefinestyle{CStyle}{
    backgroundcolor=\color{backgroundColour},   
    commentstyle=\color{mGreen},
    keywordstyle=\color{magenta},
    numberstyle=\tiny\color{mGray},
    stringstyle=\color{mPurple},
    basicstyle=\footnotesize,
    breakatwhitespace=false,         
    breaklines=true,                 
    captionpos=b,                    
    keepspaces=true,                 
    numbers=left,                    
    numbersep=5pt,                  
    showspaces=false,                
    showstringspaces=false,
    showtabs=false,                  
    tabsize=2,
    language=C
}

\makeatletter
\newcommand*\@lbracket{[}
\newcommand*\@rbracket{]}
\newcommand*\@colon{:}
\newcommand*\colorIndex{%
    \edef\@temp{\the\lst@token}%
    \ifx\@temp\@lbracket \color{black}%
    \else\ifx\@temp\@rbracket \color{black}%
    \else\ifx\@temp\@colon \color{black}%
    \else \color{vorange}%
    \fi\fi\fi
}
\makeatother
\usepackage{trace}

\setcounter{secnumdepth}{3} % To add subsubsections to the table of contents
% for QnA ref: https://latex.org/forum/viewtopic.php?t=10494
\newcounter{question}
\setcounter{question}{0}
\newcommand{\question}[1]{\item[\textbf{Q\refstepcounter{question}\thequestion.}] \textbf{#1}}
\newcommand{\answer}[1]{\item[Answer:] #1}


% A pretty common set of packages
\usepackage[margin=2.5cm]{geometry}
\usepackage[T1]{fontenc}
\usepackage[utf8]{inputenc}
\usepackage{blindtext}
\author{Maitreya Ranade}
\date{\today}
\usepackage{pdfpages}
\usepackage{graphicx}
\usepackage{amssymb}
\usepackage{amsmath}
\usepackage{color}
\usepackage{import}
\usepackage{booktabs}
\usepackage{multirow}
\usepackage{engord}
\usepackage{soul}
\usepackage{textcomp}
\usepackage{parskip}
\usepackage{setspace}
\usepackage{titlesec}
\usepackage{fancyhdr}
\usepackage{tabularx}
\usepackage{mathtools} % for '\DeclarePairedDelimiter' macro
\DeclarePairedDelimiter\norm\lVert\rVert
\usepackage{bookmark}
\pagestyle{fancy}
\usepackage[UKenglish]{babel}
\usepackage[UKenglish]{isodate}
\usepackage[skip=2pt,font=footnotesize,justification=centering]{caption}
\usepackage{natbib}
\usepackage{float}
\usepackage{dirtree}
\usepackage{url}
\usepackage{enumitem}
\usepackage{tgpagella}


% Custom environments
\newenvironment{highlight}
{\vspace{2mm} \hrule \vspace{2mm} \em}
{\vspace{2mm} \hrule \vspace{2mm}}


% Make some additional useful commands
\newcommand{\ie}{\emph{i.e.}\ }
\newcommand{\eg}{\emph{e.g.}\ }
\newcommand{\etal}{\emph{et al}}
\newcommand{\sub}[1]{$_{\textrm{#1}}$}
\newcommand{\super}[1]{$^{\textrm{#1}}$}
\newcommand{\degC}{$^{\circ}$C}
\newcommand{\wig}{$\sim$}
\newcommand{\ord}[1]{\engordnumber{#1}}
\newcommand{\num}[2]{$#1\,$#2}
\newcommand{\range}[3]{$#1$-$#2\,$#3}
\newcommand{\roughly}[2]{$\sim\!#1\,$#2}
\newcommand{\area}[3]{$#1 \! \times \! #2\,$#3}
\newcommand{\vol}[4]{$#1 \! \times \! #2 \! \times \! #3\,$#4}
\newcommand{\cube}[1]{$#1 \! \times \! #1 \! \times \! #1$}
\newcommand{\figref}[1]{Figure~\ref{#1}}
\newcommand{\eqnref}[1]{Equation~\ref{#1}}
\newcommand{\tableref}[1]{Table~\ref{#1}}
\newcommand{\secref}[1]{Section \ref{#1}}
\newcommand{\XC}{\emph{exchange-correlation}}
\newcommand{\abinit}{\emph{ab initio}}
\newcommand{\Abinit}{\emph{Ab initio}}
\newcommand{\Lonetwo}{L1$_{2}$}
\newcommand{\Dznt}{D0$_{19}$}
\newcommand{\Dtf}{D8$_{5}$}
\newcommand{\Btwo}{B$_{2}$}
\newcommand{\fcc}{\emph{fcc}}
\newcommand{\hcp}{\emph{hcp}}
\newcommand{\bcc}{\emph{bcc}}
\newcommand{\Ang}{{\AA}}
\newcommand{\inverseAng}{{\AA}$^{-1}$}
\newcommand{\comment}[1]{\textcolor{red}{[COMMENT: #1]}}
\newcommand{\more}{\textcolor{red}{[MORE]}}
\newcommand{\red}[1]{\textcolor{red}{#1}}
\newcommand{\SubItem}[1]{
    {\setlength\itemindent{15pt} \item[] \textbf{#1}}
}
% Change this to modify look of header and footer
\lhead{}
\chead{}
\rhead{}
\lfoot{}
\cfoot{\thepage{}}
\rfoot{}
\renewcommand{\headrulewidth}{0pt}
\renewcommand{\footrulewidth}{0pt}
\usepackage{hyperref}
\hypersetup{
    colorlinks=true,
    linkcolor=blue,
    filecolor=magenta,      
    urlcolor=cyan,
}
\urlstyle{same}
\newcommand{\foo}{\hspace{-2.3pt}$\bullet$ \hspace{5pt}}   % For timeline
  

\begin{document}
	\onehalfspacing
	Date: \today{} \hfill{} Name: Maitreya Ranade
	\begin{center}
		\topskip0pt
		\vspace*{\fill}
		{\LARGE Python} \\
		This report is written for personal understanding and not for publication or distribution.
		\vspace*{\fill}
	\end{center}
	
	\pagebreak
	\singlespacing
	\tableofcontents
	\pagebreak
	
    \section{Python}
    Python is an interpreted, high-level programming language, it has long attracted computational scientists and offers plenty of facilities for high performance computing.
    
    Versions:
    1.0 to 3.9.0 \\
    stable 3.7/8
    
    Python components:
    
	\begin{itemize}
        \item Data Types
        \item Containers
        \item Functions
        \item Classes
	\end{itemize}

    
    
    
    

    \subsection{Data types}
    
	\begin{itemize}
        \item Integers
        \item Floats
        \item Booleans
        \item String
	\end{itemize}
    
    
    
    

    \subsection{Containers}
    Python provides a number of container datatypes, both built-in types and those in the collections module in the Python Standard Library. Different data containers serve different purposes, provide different functionality, and present potentially very different computational performance for similar sorts of calculations. 

    list: a mutable sequence type, holding a collection of objects in a defined order (indexed by integers)
    tuple: an immutable sequence type, holding a collection of objects in a defined order (indexed by integers)
    dict: a mapping type, associating keys to values (unordered, indexed by keys)
    set: an unordered collection of unique elements (accessed through set operations)

    \subsection{Functions}
    
    \subsection{Classes}
    
    \section{References}

	\begin{itemize}
        \item \href{https://cvw.cac.cornell.edu/python}{Cornell Python Documentation}  
        \item \href{https://www.udemy.com/course/complete-python-scripting-for-automation/}{Complete Python scripting for automation}
        \item \href{https://www.udemy.com/course/complete-python-bootcamp/}{2021 Complete Python Bootcamp From Zero to Hero in Python }
        \item \href{https://www.udemy.com/course/pythonautomation/}{Learn Python: The Complete Python Automation Course!}      
	\end{itemize}
	
	
\end{document}



