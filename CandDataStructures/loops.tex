A loop statement allows the programmer to execute a statement or group of statements multiple times. C Programming languages provide various control structures that allow for more complicated execution paths.

\par In general, loops are a programming element that repeat a portion of code a set number of times until the desired process is complete. Repetitive tasks are common in programming, and loops are essential to save time and minimize errors. Loops make code more readable, manageable and organized. C language provides the following types of loops to handle looping requirements:

\subsection{while}
A while loop in C programming repeatedly executes a target statement as long as a given condition is true. It tests the condition before executing the loop body. Syntax:

\begin{lstlisting}[style=CStyle]
   while(condition) {
      statement(s);
   }
\end{lstlisting}

Here, statement(s) may be a single statement or a block of statements. The condition may be any expression, and true is any nonzero value. The loop iterates while the condition is true. When the condition becomes false, the program control passes to the line immediately following the loop.

\par Here, the key point to note is that a while loop might not execute at all. When the condition is tested and the result is false, the loop body will be skipped and the first statement after the while loop will be executed.


\subsection{for}
A for loop is a repetition control structure that allows you to efficiently write a loop that needs to execute a specific number of times. Syntax:

\begin{lstlisting}[style=CStyle]
   for ( initialization; condition; increment/decrement) {
      statement(s);
   }
\end{lstlisting}

Kindly note:
\begin{enumerate}
   \item The init step is executed first, and only once. This step allows you to declare and initialize any loop control variables. You are not required to put a statement here, as long as a semicolon appears.

   \item Next, the condition is evaluated. If it is true, the body of the loop is executed. If it is false, the body of the loop does not execute and the flow of control jumps to the next statement just after the 'for' loop.

   \item After the body of the 'for' loop executes, the flow of control jumps back up to the increment statement. This statement allows you to update any loop control variables. This statement can be left blank, as long as a semicolon appears after the condition.

   \item The condition is now evaluated again. If it is true, the loop executes and the process repeats itself (body of loop, then increment step, and then again condition). After the condition becomes false, the 'for' loop terminates.
\end{enumerate}

\subsection{do-while}
Unlike for and while loops, which test the loop condition at the top of the loop, the do-while loop in C programming checks its condition at the bottom of the loop. A do-while loop is similar to a while loop, except the fact that it is guaranteed to execute at least one time. Syntax:

\begin{lstlisting}[style=CStyle]
   do {
      statement(s);
   } while( condition );
\end{lstlisting}

Notice that the conditional expression appears at the end of the loop, so the statement(s) in the loop executes once before the condition is tested.

\par If the condition is true, the flow of control jumps back up to do, and the statement(s) in the loop executes again. This process repeats until the given condition becomes false.


\subsection{Nested loops}
C programming allows to use one loop inside another loop. A final note on loop nesting is that you can put any type of loop inside any other type of loop. For example, a 'for' loop can be inside a 'while' loop or vice versa.


\subsection{Loop Control Statements}
Loop control statements change execution from its normal sequence. When execution leaves a scope, all automatic objects that were created in that scope are destroyed. C supports the following control statements:

\subsubsection{break}
When a break statement is encountered inside a loop, the loop is immediately terminated and the program control resumes at the next statement following the loop.

\par It can also be used to terminate a case in the switch statement. If you are using nested loops, the break statement will stop the execution of the innermost loop and start executing the next line of code after the block. Syntax:

\begin{lstlisting}[style=CStyle]
   break;
\end{lstlisting}

\subsubsection{continue}
Instead of forcing termination, the continue statement forces the next iteration of the loop to take place, skipping any code in between.

For the for loop, continue statement causes the conditional test and increment portions of the loop to execute. For the while and do...while loops, continue statement causes the program control to pass to the conditional tests.
Syntax:

\begin{lstlisting}[style=CStyle]
   break;
\end{lstlisting}

\subsubsection{goto}
A goto statement in C programming provides an unconditional jump from the 'goto' to a labeled statement in the same function.

\par \textbf{NOTE:} Use of goto statement is highly discouraged in any programming language because it makes difficult to trace the control flow of a program, making the program hard to understand and hard to modify. Any program that uses a goto can be rewritten to avoid them. Syntax:

\begin{lstlisting}[style=CStyle]
   goto label;
      /* statement(s) exempted from execution */
   label: statement;
\end{lstlisting}

Here label can be any plain text except C keyword and it can be set anywhere in the C program above or below to goto statement.

\subsection{The Infinite Loop}
An infinite loop is a looping construct that does not terminate the loop and executes the loop forever. It is also called an indefinite loop or an endless loop. It either produces a continuous output or no output.

\par A loop becomes an infinite loop if a condition never becomes false. We can create an infinite loop through various loop structures like the following:

\begin{enumerate}
   \item for loop
   \item while loop
   \item do-while loop
   \item go to statement
\end{enumerate}

This is how an infinite loop can be generated from the aforementioned structures.

\begin{lstlisting}[style=CStyle]
   // using for loop
   for( ; ; ) {
      statement(s)
   }

   // using while loop
   while(1)  {  
      statement(s)
   }  

   // using do-while loop
   do  
   {  
      statement(s)
   }while(1); 
   
   // using go to statement
   label;  
   statement(s)
   goto label;  
\end{lstlisting}

\textbf{NOTE:} An infinite loop can be terminated by pressing Ctrl + C keys.