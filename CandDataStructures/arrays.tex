\begin{itemize}
    \item Constant is something that never changes. In other words, once defined cannot be modified later in the code.
    \item \#define is a preprocessor directive.
    \begin{itemize}
        \item Syntax: \#define name value (name is also called macro)
        \item Avoid semicolon at the end in the syntax.
        \item Choosing capital letters for name is good practice.
        \item Preprocessor replaces Name with value.
        \item We can use macros like functions.
        \item First expansion then evaluation.
        \item There are current predefined / standard Macros ex. \_\_TIME\_\_ \& \_\_DATE\_\_.
    \end{itemize}
    \item Const keyword
    \begin{itemize}
        \item A variable defined with the const modifier are considered variables, and not macro definitions. 
        \item syntax: const data\_type variable\_name
    \end{itemize}
    \item Const-s are handled by the compiler, where as \#define-s are handled by the pre-processor.
    \item The big advantage of const-s over \#define is type checking. \#define-s can't be type checked.
    \item Since const-s are considered variables, we can use pointers on them. This means we can typecast, move addresses, and everything else you'd be able to do with a regular variable besides change the data itself, since the data assigned to that variable is constant.
\end{itemize}