In c language, the statements are executed sequentially. Multiple times, one requires a functionality which is used multiple times in an execution. Programming languages provide various control structures that allow for more complicated execution paths.

\par Conditionals or Decision making structures require that the programmer specifies one or more conditions to be evaluated or tested by the program, along with a statement or statements to be executed if the condition is determined to be true, and optionally, other statements to be executed if the condition is determined to be false. C programming language assumes any non-zero and non-null values as true, and if it is either zero or null, then it is assumed as false value.

\par General form of a typical decision making structure found in most of the programming languages. C programming language provides the following types of Conditionals:

\subsection{if}
An if statement consists of a boolean expression followed by one or more statements. Syntax:

\begin{lstlisting}[style=CStyle]
    if(boolean_expression) {
    /* statement(s) will execute if the boolean expression is true */
    }
\end{lstlisting}

If the Boolean expression evaluates to true, then the block of code inside the 'if' statement will be executed. If the Boolean expression evaluates to false, then the first set of code after the end of the 'if' statement (after the closing curly brace) will be executed.

\subsubsection{if loop with multiple conditions}
An if statement can check multiple conditions in order to execute statements when the Boolean expression is true. Syntax:

\begin{lstlisting}[style=CStyle]
    if(boolean_expression_1 && boolean_expression_2 || (! boolean_expression_3) ) {
        /* statement(s) will execute if the boolean expression is true */
    } 
\end{lstlisting}

C allows usage of three logical operators, namely, \&\&, || and !. These are to be read as 'AND' 'OR' and 'NOT' respectively. Don't use the single symbol | and \& as they are bitwise operators. The first two operators, \&\& and ||, allow two or more conditions to be combined in an if statement. The ! operator is a NOT operator which returns 1 if a boolean expression is not true.


\subsubsection{if-else}
An if statement can be followed by an optional else statement, which executes when the Boolean expression is false. Syntax:

\begin{lstlisting}[style=CStyle]
    if(boolean_expression) {
        /* statement(s) will execute if the boolean expression is true */
    } else {
        /* statement(s) will execute if the boolean expression is false */
    }
\end{lstlisting}

If the Boolean expression evaluates to true, then the if block will be executed, otherwise, the else block will be executed.

\subsubsection{Nested if}

It is always legal in C programming to nest if-else statements, which means one can use one if or else-if statement inside another if or else-if statement(s). Syntax:

\begin{lstlisting}[style=CStyle]
    if( boolean_expression 1) {    
       /* Executes when the boolean expression 1 is true */
       if(boolean_expression 2) {
          /* Executes when the boolean expression 2 is true */
       }
    }
\end{lstlisting}

You can nest else if-else in the similar way as you have nested if statements.


\subsection{switch}
A switch statement allows a variable to be tested for equality against a list of values. Each value is called a case, and the variable being switched on is checked for each switch case. This is also a great replacement to long else if constructs. Syntax:

\begin{lstlisting}[style=CStyle]
    switch(expression) {
    
       case constant-expression  :
          statement(s);
          break; /* optional */
    	
       case constant-expression  :
          statement(s);
          break; /* optional */
      
       /* you can have any number of case statements */
       default : /* Optional */
       statement(s);
    }
\end{lstlisting}

Please note the following:
\begin{enumerate}
    \item The expression used in a switch statement must have an integral or enumerated type, or be of a class type in which the class has a single conversion function to an integral or enumerated type.

    \item You can have any number of case statements within a switch. Each case is followed by the value to be compared to and a colon. But duplicate case statements are not allowed.

    \item The constant-expression for a case must be the same data type as the variable in the switch, and it must be a constant or a literal.

    \item When the variable being switched on is equal to a case, the statements following that case will execute until a break statement is reached.

    \item When a break statement is reached, the switch terminates, and the flow of control jumps to the next line following the switch statement.

    \item Not every case needs to contain a break. If no break appears, the flow of control will fall through to subsequent cases until a break is reached.

    \item A switch statement can have an optional default case, which must appear at the end of the switch. The default case can be used for performing a task when none of the cases is true. No break is needed in the default case.
    
\end{enumerate}

\subsubsection{Nested switch}
It is possible to have a switch as a part of the statement sequence of an outer switch. Even if the case constants of the inner and outer switch contain common values, no conflicts will arise. Syntax:

\begin{lstlisting}[style=CStyle]
    switch(ch1) {
    
       case 'A': 
          printf("This A is part of outer switch" );
    
          switch(ch2) {
             case 'A':
                printf("This A is part of inner switch" );
                break;
             case 'B': /* case code */
          }
    
          break;
       case 'B': /* case code */
    }
\end{lstlisting}