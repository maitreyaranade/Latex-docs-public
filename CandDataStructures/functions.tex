\subsection{What is a Function}
A function is a self-contained block of statements that takes an input, perform a coherent task/computation and produces an output. Every C program can be thought of as a collection of these functions. 

\subsection{Need for a Function}
Why should anyone write separate functions at all:
\begin{enumerate}
    \item \textbf{Readability} Writing functions avoids rewriting the same code over and over.
    \item \textbf{Abstraction} If you are using the function in your program then you don't have to worry about how it works inside!
    \item \textbf{Reusability} Once the function is defined, it can be reused over and over again.
    \item \textbf{Modularity} Using functions it becomes easier to write programs and keep track of what they are doing. If the operation of a program can be divided into separate activities, and each activity placed in a different function, then each could be written and checked more or less independently. Separating the code into modular functions also makes the program easier to design and understand.
\end{enumerate}

\subsection{Defining a Function}
A function can also be referred as a method or a sub-routine or a procedure, etc.

\par The general form of a function definition in C programming language is as follows:
\begin{lstlisting}[style=CStyle]
    return_type function_name ( parameter list ) {
        body of the function
     }
\end{lstlisting}

A function definition in C programming consists of a function header and a function body. Here are all the parts of a function:
\begin{enumerate}
    \item Return Type: A function may return a value. The return type is the data type of the value the function returns. Some functions perform the desired operations without returning a value. 
    \item Function Name: This is the actual name of the function. The function name and the parameter list together constitute the function signature.
    \item Parameters: When a function is invoked, a value is passed to the parameter. This value is referred to as actual parameter or argument. The parameter list refers to the type, order, and number of the parameters of a function. Parameters are optional.
    \item Function Body: The function body contains a collection of statements that define what the function does.
\end{enumerate}

\subsection{Function declaration}
Function declaration is also called as function prototype represents declaring the properties of a function to the compiler. A function declaration tells the compiler about a function's name, return type, and parameters. Syntax of a function declaration is: 
\begin{lstlisting}[style=CStyle]
    return_type function_name( parameter list );
\end{lstlisting}

\subsection{Calling a Function}
After creating a C function, to use it has to be called to perform the defined task. To call a function, the required parameters along with the function name need to be passed, and the returned value has to be stored if the function has one. Syntax for calling a function is,

\begin{lstlisting}[style=CStyle]
    return_value = function_name ( parameter list );
\end{lstlisting}

\begin{highlight}
    The function which receives the control from a function call is referred as the called function.\\
    The function which transfers the control with a function call is referred as the calling function.
\end{highlight}

When a program (calling function) calls a function, the program control is transferred to the called function. A called function performs a defined task and upon reaching the end of functionality, it returns the program control back to the calling function.

\subsection{Function Arguments}
\begin{itemize}
    \item If a function is to use arguments, it must declare variables that accept the values of the arguments. 
    \item The variables declared in the function prototype or definition are known as \textbf{Formal arguments}
    \item The values that are passed to the called function from the main function are known as \textbf{Actual arguments}.
    \item The actual arguments and formal arguments must match in number, type, and order.
    \item \textbf{Formal parameters} behave like other local variables inside the function and are created upon entry into the function and destroyed upon exit.
    \item There are two methods of declaring the formal arguments: 
    \begin{enumerate}
        \item Kernighan and Ritchie (or just K \& R) method:\\ 
        function (x, y, z)\\
        int x, y, z ;
        \item ANSI method: (Commonly used)\\
        function (int x, int y, int z)
    \end{enumerate}
\end{itemize}

While calling a function, there are two ways in which arguments can be passed to a function:
\begin{enumerate}
    \item Call by value
    \item Call by reference
\end{enumerate}

\subsubsection{Call by value}
In call by value method, the value of the actual parameters is copied into the formal parameters. In other words, the value of the variable is used in the function call.

\begin{itemize}
    \item We can not modify the value of the actual parameter by the formal parameter.
    \item Different memory is allocated for actual and formal parameters since the value of the actual parameter is copied into the formal parameter.
    \item The actual parameter is the argument which is used in the function call whereas formal parameter is the argument which is used in the function definition.
\end{itemize}

\subsubsection{Call by reference}
In call by reference, the address of the variable is passed into the function call as the actual parameter. As the variables are stored in the memory, so instead of passing the value of a variable, can we not pass the location number (or address) of the variable to a function? This feature of C functions needs at least an elementary knowledge of a concept called "pointers".

\subsection{Summary and additional concepts}
\begin{enumerate}
    \item C program is a collection of one or more functions.
    \item Any C program contains at least one function.
    \item If a program contains only one function, it must be main().
    \item If a C program contains more than one function, then one (and only one) of these functions must be main(), because program execution always begins with main().
    \item There is no limit on the number of functions that might be present in a C program.
    \item A function can be called any number of times.
    \item Each function in a program is called in the sequence specified by the function calls in main().
    \item The order in which the functions are defined in a program and the order in which they get called need not necessarily be same.
    \item After each function has done its thing, control returns to main(). When main() runs out of function calls, the program ends.
    \item A function gets called when the function name is followed by a semicolon.
    \item In C language, arguments of a function are passed from right to left.
    \item A function can be called from other function, but a function cannot be defined in another function.
    \item A function can call itself. Such a process is called "recursion".
    \item The \textbf{return} statement is necessary for returning value to the calling function, if any. It serves two purposes: 
    \begin{enumerate}
        \item On executing the return statement, it immediately transfers the control back to the calling program.
        \item It returns the value present in the parentheses after return, to the calling program. In the above program the value of sum of three numbers is being returned.
    \end{enumerate}
    \item There is no restriction on the number of return statements that may be present in a function. Also, the return statement need not always be present at the end of the called function. 
    \item Whenever the control returns from a function some value is definitely returned. If a meaningful value is returned then it should be accepted in the calling program by equating the called function to some variable.
    \item If we want that a called function should not return any value, in that case, we must mention so by using the keyword void.
    \item A function can return only one value at a time.
    \item If the value of a formal argument is changed in the called function, the corresponding change does not take place in the calling function.
\end{enumerate}
    
\subsection{Types of functions}
There are two primary types of functions: 
\begin{enumerate}
    \item Library functions \eg printf(), scanf() etc.
    \item User-defined functions \eg function\_name\_1(), function\_name\_2() etc.
\end{enumerate}
As the name suggests, library functions are nothing but commonly required functions grouped together and stored in what is called a Library. This library of functions is present on the disk and is written for us by people who write compilers for us. Almost always a compiler comes with a library of standard functions. The procedure of calling both types of functions is exactly same.

\subsubsection{printf() and scanf() functions}
printf() and scanf() functions are inbuilt library functions in C programming language which are available in C library by default. These functions are declared and related macros are defined in "stdio.h" which is a header file in C language.

\subsubsection{printf}
\begin{itemize}
    \item printf() function is used to print the ("character, string, float, integer, octal and hexadecimal values") onto the output screen.
    \item syntax: printf("\%d", var)
    \item We use printf() function with \%d format specifier to display the value of an integer variable. Similarly, \%c is used to display character, \%f for float variable, \%s for string variable, \%lf for double and \%x for hexadecimal variable.
    \item To generate a newline, we use "\textbackslash n" in C printf() statement.
\end{itemize}

\subsubsection{scanf}
\begin{itemize}
    \item scanf stands for Scan Formatted string
    \item scanf() function is used to read character, string, numeric data from an input device like keyboard.
    \item syntax: scanf("\%d", \&var)
    \item The format specifier \%d is used in scanf() statement so that, the value entered is received as an integer and \%s for string.
    \item Ampersand is used before the variable name in scanf() statement to store the user input into that particular variable name as \& Variable\_name.
\end{itemize}    

\subsubsection{Points to note}
\begin{enumerate}
    \item printf() is used to display the output and scanf() is used to read the inputs.
    \item printf() and scanf() functions are declared in "stdio.h" header file in C library.
    \item All syntax in C language including printf() and scanf() functions are case sensitive.
    \item All characters in printf() and scanf() functions must be in lower case as C is a case sensitive language.
\end{enumerate}



\subsection{Static and dynamic scoping}


\iffalse
\subsection{What is a Function}
A function is a self-contained block of statements that perform a coherent task of some kind. Every C program can be thought of as a collection of these functions.

Example of a function:
\begin{lstlisting}[style=CStyle]
main()
{
message();
printf("\nCry, and you stop the monotony!");
}

message()
{
printf("\nSmile, and the world smiles with you...");
}
\end{lstlisting}
And here’s the output: 

\begin{lstlisting}[style=CStyle]
Smile, and the world smiles with you...
Cry, and you stop the monotony!
\end{lstlisting}

Here, main() itself is a function and through it we are calling the function message(). What do we mean when we say that main() "calls" the function message()? We mean that the control passes to the function message(). The activity of main() is temporarily suspended. It falls asleep while the message() function wakes up and goes to work. When the message() function runs out of statements to execute, the control returns to main(), which comes to life again and begins executing its code at the exact point where it left off. Thus, main() becomes the "calling" function, whereas message() becomes the "called" function.

\paragraph{} Calling multiple functions:
\begin{lstlisting}[style=CStyle]
main()
{
printf("\nI am in main");
italy();
brazil();
argentina();
}

italy()
{
printf("\nI am in italy");
}

brazil()
{
printf("\nI am in brazil");
}

argentina()
{
printf("\nI am in argentina");
}
\end{lstlisting}

The output of the above program when executed would be as under:
\begin{lstlisting}[style=CStyle]
I am in main
I am in italy
I am in brazil
I am in argentina
\end{lstlisting}

Summary:
\begin{enumerate}
    \item C program is a collection of one or more functions.
    \item Any C program contains at least one function.
    \item If a program contains only one function, it must be main().
    \item If a C program contains more than one function, then one (and only one) of these functions must be main(), because program execution always begins with main().
    \item There is no limit on the number of functions that might be present in a C program.
    \item A function can be called any number of times.
    \item Each function in a program is called in the sequence specified by the function calls in main().
    \item The order in which the functions are defined in a program and the order in which they get called need not necessarily be same.
    \item After each function has done its thing, control returns to main().When main() runs out of function calls, the program ends.
    \item A function gets called when the function name is followed by a semicolon. Syntax for calling a function is,
    \begin{lstlisting}[style=CStyle]
    main()
    {
    argentina();
    }
    \end{lstlisting}
    \item A function is defined when function name is followed by a pair of braces in which one or more statements may be present. Syntax for defining a function is,
    \begin{lstlisting}[style=CStyle]
    argentina()
    {
    statement 1 ;
    statement 2 ;
    statement 3 ;
    }
    \end{lstlisting}
    \item A function can be called from other function, but a function cannot be defined in another function.
    \item A function can call itself. Such a process is called "recursion".
    \item There are basically two types of functions: 
    \begin{enumerate}
        \item Library functions Ex. printf(), scanf() etc.
        \item User-defined functions Ex. argentina(), brazil() etc.
    \end{enumerate}
    As the name suggests, library functions are nothing but commonly required functions grouped together and stored in what is called a Library. This library of functions is present on the disk and is written for us by people who write compilers for us. Almost always a compiler comes with a library of standard functions. The procedure of calling both types of functions is exactly same.
\end{enumerate}
    
\subsubsection{printf() and scanf() functions}
printf() and scanf() functions are inbuilt library functions in C programming language which are available in C library by default. These functions are declared and related macros are defined in "stdio.h" which is a header file in C language.

\subsubsection{printf}
\begin{itemize}
    \item printf() function is used to print the ("character, string, float, integer, octal and hexadecimal values") onto the output screen.
    \item syntax: printf("\%d", var)
    \item We use printf() function with \%d format specifier to display the value of an integer variable. Similarly, \%c is used to display character, \%f for float variable, \%s for string variable, \%lf for double and \%x for hexadecimal variable.
    \item To generate a newline, we use "\textbackslash n" in C printf() statement.
\end{itemize}

\subsubsection{scanf}
\begin{itemize}
    \item scanf stands for Scan Formatted string
    \item scanf() function is used to read character, string, numeric data from an input device like keyboard.
    \item syntax: scanf("\%d", \&var)
    \item The format specifier \%d is used in scanf() statement so that, the value entered is received as an integer and \%s for string.
    \item Ampersand is used before the variable name in scanf() statement to store the user input into that particular variable name as \& Variable\_name.
\end{itemize}    

\subsubsection{Note}
\begin{enumerate}
    \item printf() is used to display the output and scanf() is used to read the inputs.
    \item printf() and scanf() functions are declared in "stdio.h" header file in C library.
    \item All syntax in C language including printf() and scanf() functions are case sensitive.
    \item All characters in printf() and scanf() functions must be in lower case as C is a case sensitive language.
\end{enumerate}


\subsection{Why Use Functions}
Why should anyone write separate functions at all:
\begin{enumerate}
    \item \textbf{Readability} Writing functions avoids rewriting the same code over and over.
    \item \textbf{Abstraction} If you are using the function in your program then you don't have to worry about how it works inside!
    \item \textbf{Reusability} Once the function is defined, it can be reused over and over again.
    \item \textbf{Modularity} Using functions it becomes easier to write programs and keep track of what they are doing. If the operation of a program can be divided into separate activities, and each activity placed in a different function, then each could be written and checked more or less independently. Separating the code into modular functions also makes the program easier to design and understand.
\end{enumerate}

\subsection{Passing Values between Functions}
The 'calling' function has to communicate certain information to the 'called' function. The mechanism used to convey information to the function is the 'argument'. The arguments are sometimes also called 'parameters'.

Consider the following program. In this program, in main()we receive the values of a, b and c through the keyboard and then output the sum of a, b and c. However, the calculation of sum is done in a different function called calsum(). If sum is to be calculated in calsum()and values of a, b and c are received in main(), then we must pass on these values to calsum(), and once calsum()calculates the sum we must return it from calsum()back to main().

\begin{lstlisting}[style=CStyle]
/* Sending and receiving values between functions */
main()
{
    int a, b, c, sum ;
    printf("\nEnter any three numbers ");
    scanf("%d %d %d", &a, &b, &c);
    sum = calsum(a, b, c);
    printf("\nSum = %d", sum);
}

calsum(x, y, z)
int x, y, z ;
{
    int d ;
    d = x + y + z ;
    return(d);
}
\end{lstlisting}

And here is the output:
\begin{lstlisting}[style=CStyle]
Enter any three numbers 10 20 30
Sum = 60
\end{lstlisting}


There are a number of things to note about this program:
\begin{enumerate}
    \item In this program, from the function main() the values of a, b and c are passed on to the function calsum(), by making a call to the function calsum().
    \item In the calsum() function these values get collected in three variables x, y and z.
    \item The variables a, b and c are called 'actual arguments', whereas the variables x, y and z are called 'formal arguments'. \item Any number of arguments can be passed to a function being called. However, the type, order and number of the actual and formal arguments must always be same. 
    \item There are two methods of declaring the formal arguments. 
    \begin{enumerate}
        \item Kernighan and Ritchie (or just K \& R) method: (used in this program): \\ calsum(x, y, z)\\
        int x, y, z ;
        \item ANSI method: (Commonly used)\\
        calsum(int x, int y, int z)
    \end{enumerate}
    \item The \textbf{return} statement is necessary for returning value to the calling function, if any. It serves two purposes: \begin{enumerate}
        \item On executing the return statement, it immediately transfers the control back to the calling program.
        \item It returns the value present in the parentheses after return, to the calling program. In the above program the value of sum of three numbers is being returned.
    \end{enumerate}
    \item There is no restriction on the number of return statements that may be present in a function. Also, the return statement need not always be present at the end of the called function. 
    \item Whenever the control returns from a function some value is definitely returned. If a meaningful value is returned then it should be accepted in the calling program by equating the called function to some variable.
    \item If we want that a called function should not return any value, in that case, we must mention so by using the keyword void.
    \item A function can return only one value at a time.
    \item If the value of a formal argument is changed in the called function, the corresponding change does not take place in the calling function.
\end{enumerate}

\subsection{Scope Rule of Functions}
Kindly refer to the Variables section.

\subsection{Calling Convention}
Calling convention indicates the order in which arguments are passed to a function when a function call is encountered. There are two possibilities here:
\begin{enumerate}
    \item Arguments might be passed from left to right.
    \item Arguments might be passed from right to left. (Followed by C)
\end{enumerate}
The order of passing arguments becomes an important consideration. For example:

\begin{lstlisting}[style=CStyle]
int a = 1;
printf("%d %d %d", a, ++a, a++);
\end{lstlisting}
It appears that this printf() would output 1 2 3.
This however is not the case. Surprisingly, it outputs 3 3 1. As C's calling convention is from right to left, i.e. firstly 1 is passed through the expression a++ and then a is incremented to 2. Then result of ++a is passed. That is, a is incremented to 3 and then passed. Finally, latest value of a, i.e. 3, is passed. Thus in right to left order 1, 3, 3 get passed. Once printf() collects them it prints them in the order in which we have asked it to get them printed (and not the order in which they were passed). Thus 3 3 1 gets printed.


\subsection{Function Declaration and Prototypes}
Any C function by default returns an \textbf{int} value. More specifically, whenever a call is made to a function, the compiler assumes that this function would return a value of the type \textbf{int}. If we desire that a function should return a value other than an \textbf{int}, then it is necessary to explicitly mention so in the calling function as well as in the called function. What it means is square() is a function that receives a float and returns a float. For example,\\
float square (float);\\
This statement is often called the prototype declaration of the square() function.

\subsection{Call by Value and Call by Reference}
There are two methods to pass the data into the function in C language, 
\begin{enumerate}
    \item Call by value
    \item Call by reference
\end{enumerate}

\subsubsection{Call by value}
In call by value method, the value of the actual parameters is copied into the formal parameters. In other words, the value of the variable is used in the function call.

\begin{itemize}
    \item We can not modify the value of the actual parameter by the formal parameter.
    \item Different memory is allocated for actual and formal parameters since the value of the actual parameter is copied into the formal parameter.
    \item The actual parameter is the argument which is used in the function call whereas formal parameter is the argument which is used in the function definition.
\end{itemize}

\subsubsection{Call by reference}
In call by reference, the address of the variable is passed into the function call as the actual parameter. As the variables are stored in the memory, so instead of passing the value of a variable, can we not pass the location number (or address) of the variable to a function? This feature of C functions needs at least an elementary knowledge of a concept called "pointers".

\fi